\begin{abstract}{english}

Graph based models are used for tasks with increasing size and computational demands. The paper focuses on leveraging methods for pretraining on coarser graphs with HARP as the method of choice. The method is generalized using partially injective homomorphisms, a concept from the field of data mining. Such a way of producing graph coarsenings is shown to be feasible and not to affect the performance of HARP in a negative way. Also, the performance-complexity characterics of these methods are studied and HARP is established as a way of efficient pretraining which can reduce the ammount of computational power needed to train graph-based models on large data.

\end{abstract}

\begin{abstract}{czech}

\begin{czech}
Výpočetní modely založené na grafech se v současné době používají na data se stále se zvětšující velikostí a tedy i nároky na výpočetní výkon. Tento článek se zaměřuje na metody pro pretraining na hrubších grafech, konkrétně na metodu HARP. Tato metoda je zobecněna pomocí částečně injektivních homomorfismů, konceptu z oboru dolování z dat. Je ukázáno, že takový způsob generování hrubších grafů je prakticky použitelný a nijak negativně neovlivňuje výkonnost metody HARP. Poměr výkonnosti a výpočetní náročnosti byl také vyhodnocen a metoda HARP byla prokázána jako efektivní způsob pretrainingu, který může zmenšit výpočetní nároky modelů založených na velkých grafech.
\end{czech}

\end{abstract}
