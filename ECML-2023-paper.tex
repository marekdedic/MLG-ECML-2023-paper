% !TEX TS-program = pdflatex

%%%%%%%%%%%%%%%%%%%%%%%%%%%%%%%%%%%%%%%%%%%%%%%%%%%%%%%%%%%%%%%%%%%%%
%%                                                                 %%
%% Please do not use \input{...} to include other tex files.       %%
%% Submit your LaTeX manuscript as one .tex document.              %%
%%                                                                 %%
%% All additional figures and files should be attached             %%
%% separately and not embedded in the \TeX\ document itself.       %%
%%                                                                 %%
%%%%%%%%%%%%%%%%%%%%%%%%%%%%%%%%%%%%%%%%%%%%%%%%%%%%%%%%%%%%%%%%%%%%%

\documentclass[sn-mathphys,pdflatex,iicol]{sn-jnl}% Math and Physical Sciences Reference Style

%%%% Standard Packages
%\usepackage{polyglossia} % Must come before biblatex

\usepackage{bm}
\usepackage{csquotes}
%\usepackage{fontspec}
\usepackage{hyperref}
%\usepackage{lua-visual-debug}
\usepackage{tabularx}
%%%%

\jyear{2022}

%% as per the requirement new theorem styles can be included as shown below
%%\theoremstyle{thmstyleone}%
%%\newtheorem{theorem}{Theorem}[section]% meant for sectionwise numbers
%% optional argument [theorem] produces theorem numbering sequence instead of independent numbers for Proposition
%%\newtheorem{proposition}[theorem]{Proposition}%
%%\newtheorem{proposition}{Proposition}% to get separate numbers for theorem and proposition etc.

%%\theoremstyle{thmstyletwo}%
%%\newtheorem{example}{Example}%
%%\newtheorem{remark}{Remark}%

%%\theoremstyle{thmstylethree}%
%%\newtheorem{definition}{Definition}%

\raggedbottom
%%\unnumbered% uncomment this for unnumbered level heads

%\setdefaultlanguage{english}
%\setotherlanguage{czech}

\hypersetup{
	pdfencoding=auto,
	unicode=true,
	bookmarksopen=true,
	bookmarksopenlevel=3
}

\newcommand{\name}[1]{\textit{#1}}
\newcommand{\mathfield}{\ensuremath{\mathbb}}
\newcommand{\mathmat}{\ensuremath{\mathbf}}
\newcommand{\mathset}{\ensuremath{\mathbb}}
\newcommand{\mathspace}{\ensuremath{\mathcal}}
\newcommand{\mathvec}{\ensuremath{\bm}}

\newcounter{enumroman}
\newenvironment{romanitems}{\begin{list}{\bfseries(\roman{enumroman})\hfill}{\usecounter{enumroman}\setlength{\labelwidth}{\leftmargin}\addtolength{\labelwidth}{-1\labelsep}\topsep=0mm plus 2pt\itemsep=0mm\parsep=0mm plus 2pt\itemindent=0mm}}{\end{list}}

\DeclareMathOperator*{\argmin}{arg\,min}
\DeclareMathOperator*{\argmax}{arg\,max}

\begin{document}

\title[Adaptive graph coarsening]{Balancing performance and complexity with adaptive graph coarsening}

\author*[1,2]{\fnm{Marek} \sur{Dědič}}\email{marek@dedic.eu}

\author[2]{\fnm{Lukáš} \sur{Bajer}}\email{lubajer@cisco.com}

\author[2]{\fnm{Pavel} \sur{Procházka}}\email{paprocha@cisco.com}
\author[3]{\fnm{Martin} \sur{Holeňa}}\email{martin@cs.cas.cz}

\affil[1]{\orgdiv{Faculty of Nuclear Sciences and Physical Engineering}, \orgname{Czech Technical University in Prague}, \orgaddress{\street{Břehová 7}, \city{Prague}, \postcode{110 00}, \country{Czech Republic}}}

\affil[2]{\orgdiv{Cognitive Intelligence}, \orgname{Cisco Systems, Inc.}, \orgaddress{\street{Karlovo náměstí 10}, \city{Prague}, \postcode{120 00}, \country{Czech Republic}}}

\affil[3]{\orgdiv{Institute of Computer Science}, \orgname{Czech Academy of Sciences}, \orgaddress{\street{Pod vodárenskou věží 2}, \city{Prague}, \postcode{182 07}, \country{Czech Republic}}}

\begin{abstract}
  Graph based models are used for tasks with increasing size and computational demands. We present a method for node classification that allows a user to precisely select the resolution at which the graph in question should be pretrained. Our method builds on an existing algorithm for pretraining on coarser graphs, HARP. We extend both main parts of the algorithm in order to tune the effect of graph coarsening on the accuracy of node classification on a fine level. We present a framework for graph coarsening, covering, apart from HARP, two alternative algorithms based on graph diffusion convolution. Moreover, we present a novel way for refining the reduced graph in a targeted way based on the node classification confidence of particular nodes. Together, these enhancements provide sufficient detail where needed, while collapsing structures where per-node information is not necessary for sufficient node classification accuracy. Hence, the method provides a meta-model for enhancing graph embedding models such as node2vec. We apply it to several datasets, compare the considered coarsenings on them and discuss the differing behaviour on each of them in the context of their properties.

  \keywords{
    Graph representation learning \and
    Graph coarsening \and
    Graph diffusion \and
    Graph homophily \and
    Performance-complexity trade-off \and
    HARP
  }
\end{abstract}


\maketitle

\section{Introduction}
Across a wide variety of applications and domains, graphs emerge as a domain-independent and ubiquitous way of organizing structured data. Consequently, machine learning on graphs has, in recent years, seen an explosion in popularity, breadth and depth of both research and applications. While there have been significant advances in algorithms for learning from graph data \cite{defferrard_convolutional_2016,kipf_semi-supervised_2017}, the underlying graph topology has, until recent works \cite{topping_understanding_2021,velickovic_geometric_2021}, received much less attention. In the reported research, we investigate the interplay of graph coarsening and the quality of its learned embedding (as studied, for example, by \cite{akyildiz_understanding_2020,makarov_survey_2021}), which in turn entails an interplay between the coarsening and the performance of a downstream task, in our case, node classification.

The main aim of this work is to explore the performance-complexity characteristics in the context of graph learning, as introduced in \cite{prochazka_downstream_2022}. Consider an undirected graph \( G \) with nodes \( V \left( G \right) \) and edges \( E \left( G \right) \). The result of a repeated application of graph coarsening is a sequence of graphs \( G_0, G_1, G_2, \dots, G_L \) where \( G_0 = G \).
Given a model \( M \) that operates on graphs, a performance metric, and a complexity metric, the sequence \( G_0, G_1, \dots, G_L \) corresponds to points in the performance-complexity plane, where advancing along the sequence generally hurts performance and decreases complexity.

This performance-complexity characteristic allows for a choice of a \textbf{working point} that is optimal for the particular use-case. The choice of the working point, suitable performance metric and complexity metric are subjective and depend on the particular use-case, downstream task and the environment in which the model is to be deployed. In this work, the transductive node classification accuracy on a testing dataset is chosen as the performance metric. For the complexity metric, the number of nodes in the graph was chosen as it constitutes a good proxy for real-world algorithmic complexity, as shown in \cite{chiang_cluster-gcn_2019}.

While the methods proposed in the rest of this work may yield models and graphs with lower computational demands than models using the original graph, the algorithm for finding the optimal working point itself entails running the same complex models on multiple graphs, therefore potentially offsetting any gains from the lower complexity of the model itself. To overcome these potential shortcomings, the following options are considered:
\begin{itemize}
  \item The optimal working point may generalize to datasets other than the one used for the performance-complexity analysis, for example when collecting data from the same source periodically.
  \item The whole performance-complexity curve is not needed to choose the optimal working point. In the context of this work, the graphs are evaluated in reverse order, i.e. starting with \( G_L \). As such, the evaluation only needs to be run until reaching a working point that is acceptable for the intended use-case.
\end{itemize}
Further discussion of the performance-complexity trade-off problem is considered in \cite{prochazka_downstream_2022}.

\section{Related work}

The publication most relevant to our research is \cite{chen_harp_2018}, in which the HARP approach is proposed. Because we directly extend and modify this method, it will be recalled in some detail in Section \ref{sec:harp}. Other important works concerning graph coarsening are \cite{akyildiz_understanding_2020,cai_graph_2022,chen_graph_2022}, which survey numerous coarsening methods, \cite{huang_scaling_2021}, which presents results concerning scalability of graph coarsening, \cite{catalyurek_multithreaded_2012,herrmann_multilevel_2019}, which establish coarsening as a basis for partitioning, and \cite{loukas_graph_2019}, which shows relationships of graph coarsening to properties of the Laplacian. In view of the fact that the HARP approach is a multilevel approach, we paid attention also to the multilevel graph coarsening methods proposed in \cite{bethune_hierarchical_2020,liu_hierarchical_2021,xie_graph_2020,zhang_harp_2021}, among them \cite{zhang_harp_2021} also being inspired by HARP.

In a broader context, our research is related to the more general topic of graph reduction, which apart from graph coarsening includes also graph sparsification and condensation. A general framework covering both coarsening and sparsification has been proposed in \cite{bravo_hermsdorff_unifying_2019}. Graph condensation is a more recent approach \cite{jin_condensing_2022,jin_graph_2022}, inspired by the gradient matching scheme for the construction of training datasets, called dataset condensation \cite{zhao_dataset_2021}: To a given original graph, it attempts to construct a much smaller synthetic graph such that the gradients of the parameters of a considered graph neural network with respect to both graphs match. Also of note is the recent work \cite{kammer_space-efficient_2022}, presenting an alternative coarsening approach for planar graphs and \cite{liu_comprehensive_2022}, which sparsifies not only the graph topology, but simultaneously also the features of its nodes and weights of graph neural network used for its embedding. Elaboration of graph coarsening methods in machine learning can build on several decades of their successful application, such as pairwise aggregation, independent sets, or algebraic distance, in numerical linear algebra \cite{chen_graph_2022}, including in particular multilevel graph coarsening \cite{osei-kuffuor_matrix_2015,ubaru_sampling_2019}.

More recently, a connection of graph coarsening with another more general topic has been addressed, namely with pooling in graph neural networks. In a framework presented in \cite{grattarola_understanding_2022}, pooling is viewed as a composition of three subsequent transformations: selection, in which the nodes of the input graph are mapped to the nodes of the pooled one, reduction, in which the node attributes of the input graph are aggregated into the node attributes of the pooled one, and connection, in which the edges and possibly edge attributes of the input graph are mapped to the edges and possibly edge attributes of the pooled one. This sequence of transformations is perfectly relevant also to coarsening methods, as was in \cite{grattarola_understanding_2022} demonstrated for the methods NMF \cite{bacciu_non-negative_2019} and top-K \cite{cangea_towards_2018,gao_graph_2019}.

\section{The performance-complexity trade-off problem}\label{sec:performance-complexity}

The main aim of this work is to explore the performance-complexity characteristics in the context of graph learning, as introduced in \cite{prochazka_downstream_2022}. Consider an undirected graph \( G \) with nodes \( V \left( G \right) \) and edges \( E \left( G \right) \). The result of a repeated application of graph coarsening is a sequence of graphs \( G_0, G_1, G_2, \dots, G_L \) where \( G_0 = G \).
Given a model \( M \) that operates on graphs, a performance metric, and a complexity metric, the sequence \( G_0, G_1, \dots, G_L \) corresponds to points in the performance-complexity plane, where advancing along the sequence generally hurts performance and decreases complexity.

This performance-complexity characteristic allows for a choice of a \textbf{working point} that is optimal for the particular use-case. In this work, the transductive node classification accuracy on a testing dataset is chosen as the performance metric. For the complexity metric, the number of nodes in the graph was chosen as it constitutes a good proxy for real-world algorithmic complexity, as shown in \cite{chiang_cluster-gcn_2019}. The methods proposed in the rest of this work evaluates the graphs in reverse order, i.e. starting with the simplest one. Due to this, the algorithm allows for lower complexity of selecting the working point, despite repeated evalueations of the model. Further discussion of the performance-complexity trade-off problem is considered in \cite{prochazka_downstream_2022}.

\section{An overview of the HARP algorithm}\label{sec:harp}

Our work builds on the HARP method \cite{chen_harp_2018} for pretraining methods such as node2vec \cite{grover_node2vec_2016} on coarsened graphs. While HARP itself works with and modifies the graph structure, this is not the main interest of its authors, who focus more on the classification accuracy for the original graph. The sequence \( G_0, G_1, G_2, \dots, G_L \) is generated in HARP consecutively. Let \( \varphi_i \) denote this mapping \( G_i = \varphi_i \left( G_{i - 1} \right) \). Following \cite{schulz_mining_2019}, we restrict the definition of such a coarsening \( \varphi_i \) to only consist of a series of edge contractions \( \mathcal{C} \subseteq E \left( G \right) \). In an overview, the HARP algorithm first ahead-of-time consecutively coarsens the graph. The method itself can then be executed by repeating the following steps on the graphs from the coarsest to the finest (i.e. from \( G_L \) to \( G_0 \)):
\begin{enumerate}
  \item \textbf{Training on an intermediary graph}. The graph embedding model is trained on \( G_i \), producing its embedding \( \Phi_{G_i} \).
  \item \textbf{Embedding prolongation}. The embedding \( \Phi_{G_i} \) is \textit{prolonged} (i.e. refined) into \( \Phi_{G_{i - 1}} \) by copying embeddings of merged nodes. \( \Phi_{G_{i - 1}} \) is then used as the starting point for training on \( G_{i - 1} \).
\end{enumerate}
The particular details of the coarsening and prolongation steps are further explained in \cite{chen_harp_2018}.

\section{HARP extension for flexible performance-complexity balancing}\label{sec:our-method}

Graph representation learning methods such as node2vec typically have a large number of parameters -- on the widely used OGBN-ArXiv dataset (see \cite{hu_open_2021}), the state-of-the-art node2vec model has over 21 million parameters. At the same time, recent works in the domain of graph learning have started to focus more heavily on simpler methods as a competitive alternative to heavy-weight ones (see \cite{frasca_sign_2020,huang_combining_2020}). As the authors of \cite{chen_harp_2018} observed, HARP improves the performance of models when fewer labelled data are available. The proposed lower complexity models based on HARP could also improve performance in a setting where only low fidelity data are available for large parts of the graph. Coarser models could be trained on them, with a subsequent training of finer models using only a limited sample of high fidelity data.

While the prolongation used by HARP is sufficient when used only as a means of pre-training, the approach is far too crude when studying the relationship between graph complexity and the quality of graph embedding as a single coarsening iteration can reduce the number of nodes to less than half. In order to overcome this limitation, we present the adaptive prolongation approach, which aims to replace the fixed steps defined by the used coarsening algorithm (such as HARP) by a variable number of smaller \enquote{micro-steps}, each of a predefined size that can be chosen independently from the underlying coarsening and its step size. The \( L \) coarsening steps are thus decoupled from \( K \) prolongation steps, where \( K \) is independent of \( L \). The prolongation steps are driven by the interplay of the downstream task with the local properties of the underlying graph, enabling the method to produce embeddings with different level of granularity in different parts of the graph, e.g. an embedding that is coarse inside clusters of similar nodes and at the same time fine at the border between such clusters.

\begin{figure}
	\centering
	\includegraphics[width=\textwidth]{images/adaptive-prolongation/adaptive-prolongation.pdf}
	\caption{A schematic explanation of the adaptive prolongation algorithm for obtaining the embedding \( \Psi_{i} \) from \( \Psi_{i + 1} \).}
	\label{fig:adaptive-prolongation}
\end{figure}

Let us denote \( \Psi_K, \dots, \Psi_0 \) the resulting embedding sequence. Similarly to standard HARP prolongation, the algorithm starts with the coarsest graph \( G_L \), trains a graph model to compute its embedding \( \Psi_K \) and gradually refines it until reaching the embedding \( \Psi_0 \). These prolongation steps are interlaid with continued training of the graph model, as in standard HARP. A description of a single prolongation step from \( \Psi_{i + 1} \) to \( \Psi_i \) follows and is schematically outlined in Figure~\ref{fig:adaptive-prolongation}.

The procedure keeps track of all the edge contractions that were made in the dataset augmentation part of the algorithm and gradually reverses them. To this end, apart from the embedding \( \Psi_i \), the set of all contractions yet to be reversed as of step \( i \) is kept as \( \mathcal{C}_L^{(i)}, \dots, \mathcal{C}_0^{(i)} \), with the initial values \( \mathcal{C}_j^{(K)} \) corresponding to the underlying coarsening \( \varphi_j \) as defined in Section~\ref{sec:harp}.

In each prolongation step, the embedding \( \Psi_{i + 1} \) is prolonged to \( \Psi_i \) by selecting a set of \( n_p \) contractions \( \mathcal{C}_\mathrm{prolong} \) and undoing them by copying and reusing the embedding of the node resulting from the contraction to both of the contracted nodes. To obtain \( \mathcal{C}_\mathrm{prolong} \), nodes of \( G_0 \) are first ordered in such a way that corresponds to the usefulness of prolonging them. Subsequently, the set \( \mathcal{C}_\mathrm{prolong} \) is selected from \( \mathcal{C}_L^{(i + 1)}, \dots, \mathcal{C}_0^{(i + 1)} \) by selecting contractions affecting nodes in the aforementioned order, until \( n_p \) contractions are selected. If multiple contractions affecting the same node are available in the sequence \( \mathcal{C}_L^{(i + 1)}, \dots, \mathcal{C}_0^{(i + 1)} \), one is selected from \( \mathcal{C}_j^{(i + 1)} \) corresponding to the coarsest-level coarsening. The sequence \( \mathcal{C}_L^{(i)}, \dots, \mathcal{C}_0^{(i)} \) is produced from \( \mathcal{C}_L^{(i + 1)}, \dots, \mathcal{C}_0^{(i + 1)} \) by removing all of the edges contained in \( \mathcal{C}_\mathrm{prolong} \).

To obtain an ordering of nodes of \( G_0 \) based on the usefulness of their prolongation, the embedding \( \Psi_{i + 1} \) is fully prolonged to a temporary embedding of the full graph, \( \Psi_0^\mathrm{temp} \). The downstream model is then trained using this temporary embedding to obtain \( \mathmat{Y}_\mathrm{pred} \), the predicted posterior distribution of classes for each node in \( G_0 \) (e.g. the output of the softmax layer of an MLP). The entropy of this distribution is measured, representing the amount of uncertainty in the classifier for each given node. The nodes are ordered based on the entropy from highest to lowest. This reflects the principle that it is most useful to prolong those nodes where the downstream classifier is the least certain. For downstream tasks other than node classification, the ordering would need to be defined in a different manner (for example using labels, which are not available for all nodes in our case), however the approach of prolonging the nodes about which the downstream model is the most uncertain can be extended to other tasks.

\section{Experimental evaluation}\label{sec:experimental-evaluation}

The proposed methods were experimentally verified on 10 pulicly available datasets: Cora, CiteSeer \cite{yang_revisiting_2016}, Twitch variants DE and EN \cite{rozemberczki_multi-scale_2021}, PubMed \cite{yang_revisiting_2016}, DBLP \cite{bojchevski_deep_2018}, IMDB \cite{fu_magnn_2020}, both variants of the Coauthor dataset \cite{shchur_pitfalls_2019} and the OGB ArXiv dataset \cite{hu_open_2021}.

The achitecture of the model was identical accross the datasets, with the only difference being in the values of the hyper-parameters,  which were initially set to values used in prior art (see \cite{hu_open_2021, fey_fast_2019}) and then manually fine-tuned separately for each dataset. The node2vec algorithm was used for generating the node embeddings, with an MLP classifier providing the predictions for node classification. The experiment was run \( 10 \) times end-to-end and results averaged.

In order to study the effect of the adaptive prolongation, the adaptive prolongation method was used to assess the performance of downstream transductive classification at different coarsening levels. The previously described model was trained with adaptive prolongation based on coarsenings pre-computed by the HARP coarsening algorithm. For each prolongation step, the intermediary embedding was afterwards fully prolonged to obtain an embedding of the original graph \( G \). A classifier was then trained with this embedding as input. This setup allows us to compare classification accuracy at each step of the adaptive prolongation. Figure~\ref{fig:adaptive-coarsening} shows the results of this experiment, compared with a baseline node2vec model (that is, without any coarsening or prolongation) that was trained for the same number of epochs as the total epochs of the adaptive model over all prolongation steps.

\begin{figure}
  \centering
  \includegraphics[width = \linewidth]{images/adaptive-coarsening/adaptive-coarsening.pdf}
  \caption{Downstream classifier accuracies at different steps of adaptive prolongation. Dashed line shows the baseline node2vec model accuracy. The node count is taken relative to the total node count in each dataset. The results are averaged over multiple runs, with the solid line representing the mean and the shaded area denoting one standard deviation.}
  \label{fig:adaptive-coarsening}
\end{figure}

Following recent best-practice recommendations regarding verifying the statistical validity of results \cite{benavoli_time_2017}, the results were studied from the point of view of Bayesian estimation. The performance of the model was compared to that of the baseline model at \( k \)-th deciles of the node count, for all possible values of \( k \). The comparison was done using the Bayesian Wilcoxon signed-rank test \cite{benavoli_bayesian_2014} for 3 different widths of the region of practical equivalence (ROPE), 1\%, 5\% and 10\%. The probabilities that the two models are practically equivalent are listed in Table~\ref{tab:bayesian-adaptive}. Of a particular note is the fact that at 60\% complexity, the models have over a 99\% probability of being within 10 percentage points of performance -- showing that the proposed method may offer a significant complexity reduction in exchange for a relatively minor decrease in performance.

\begin{table}
  \caption{The probabilities that the adaptive approach will be practically equivalent to node2vec when compared on different fractions of the full graph and with different widths of the region of practical equivalence.}
  \label{tab:bayesian-adaptive}
  \centering
  \begin{tabular}{lrrr}
    \toprule
    \textbf{Nodes} & \textbf{1\% ROPE} & \textbf{5\% ROPE} & \textbf{10\% ROPE} \\
    \midrule
    \textbf{10\%}  & 0\%               & 0.3\%             & 2.5\%              \\
    \textbf{20\%}  & 0\%               & 0.8\%             & 14.1\%             \\
    \textbf{30\%}  & 0\%               & 1.7\%             & 35.3\%             \\
    \textbf{40\%}  & 0\%               & 5.3\%             & 72.0\%             \\
    \textbf{50\%}  & 0.1\%             & 35.3\%            & 85.7\%             \\
    \textbf{60\%}  & 0.6\%             & 62.2\%            & 99.7\%             \\
    \textbf{70\%}  & 32.0\%            & 84.7\%            & 100.0\%            \\
    \textbf{80\%}  & 30.0\%            & 99.9\%            & 100.0\%            \\
    \textbf{90\%}  & 48.9\%            & 100.0\%           & 100.0\%            \\
    \textbf{100\%} & 87.7\%            & 100.0\%           & 100.0\%            \\
    \bottomrule
  \end{tabular}
\end{table}

\section{Conclusion}

In this work, an extension of the HARP algorithm was proposed, which generalizes it from a method for pretraining to a general graph reduction framework. A novel approach to prolonging graphs in the HARP setting was presented that selectively prolongs the graph in a way that maximizes performance of the considered downstream task under limited graph size. All of the proposed methods were experimentally verified, with the headline result being that at about 40\% reduction in node count, the accuracy was still reasonably close to the accuracy on a full graph for most datasets. In future work, a direct way of tackling the outlined problem without the double procedure of coarsening and prolongation may be studied as an alternative to the proposed approach.


\section*{Declarations}

\subsection*{Funding}

The research reported in this paper has been supported by the German Research Foundation (DFG) funded project 467401796.

Marek Dědič, Lukáš Bajer and Pavel Procházka are employed by Cisco Systems, Inc. Martin Holeňa is employed by the Institute of Computer Science of the Czech Academy of Sciences, the Czech Technical University in Prague and the University of Rostock.

\subsection*{Conflicts of interest/Competing interests}

Martin Holeňa is a reviewer for the 2023 ECML journal track.

\subsection*{Ethics approval}

Not applicable.

\subsection*{Consent to participate}

Not applicable.

\subsection*{Consent for publication}

Not applicable.

\subsection*{Availability of data and material}

All used datasets are publicly available.


\subsection*{Authors' contributions}

All authors contributed to the research conception and design. Experiment programming and execution was done by Marek Dědič. Statistical interpretation of the results was done by Marek Dědič and Martin Holeňa. The first draft of the manuscript was written by Marek Dědič and all authors commented on previous versions of the manuscript. All authors read and approved the final manuscript.

\bibliography{zotero}

\end{document}
