\section{Introduction}
Across a wide variety of applications and domains, graphs emerge as a domain-independent and ubiquitous way of organizing structured data. Consequently, machine learning on graphs has, in recent years, seen an explosion in popularity, breadth and depth of both research and applications. While there have been significant advances in algorithms for learning from graph data \cite{defferrard_convolutional_2016,kipf_semi-supervised_2017}, the underlying graph topology has, until recent works \cite{topping_understanding_2021,velickovic_geometric_2021}, received much less attention. In the reported research, we investigate the interplay of graph coarsening and the quality of its learned embedding (as studied, for example, by \cite{akyildiz_understanding_2020,makarov_survey_2021}), which in turn entails an interplay between the coarsening and the performance of a downstream task, in our case, node classification.

The main aim of this work is to explore the performance-complexity characteristics in the context of graph learning, as introduced in \cite{prochazka_downstream_2022}. Consider an undirected graph \( G \) with nodes \( V \left( G \right) \) and edges \( E \left( G \right) \). The result of a repeated application of graph coarsening is a sequence of graphs \( G_0, G_1, G_2, \dots, G_L \) where \( G_0 = G \).
Given a model \( M \) that operates on graphs, a performance metric, and a complexity metric, the sequence \( G_0, G_1, \dots, G_L \) corresponds to points in the performance-complexity plane, where advancing along the sequence generally hurts performance and decreases complexity.

This performance-complexity characteristic allows for a choice of a \textbf{working point} that is optimal for the particular use-case. The choice of the working point, suitable performance metric and complexity metric are subjective and depend on the particular use-case, downstream task and the environment in which the model is to be deployed. In this work, the transductive node classification accuracy on a testing dataset is chosen as the performance metric. For the complexity metric, the number of nodes in the graph was chosen as it constitutes a good proxy for real-world algorithmic complexity, as shown in \cite{chiang_cluster-gcn_2019}.

While the methods proposed in the rest of this work may yield models and graphs with lower computational demands than models using the original graph, the algorithm for finding the optimal working point itself entails running the same complex models on multiple graphs, therefore potentially offsetting any gains from the lower complexity of the model itself. To overcome these potential shortcomings, the following options are considered:
\begin{itemize}
  \item The optimal working point may generalize to datasets other than the one used for the performance-complexity analysis, for example when collecting data from the same source periodically.
  \item The whole performance-complexity curve is not needed to choose the optimal working point. In the context of this work, the graphs are evaluated in reverse order, i.e. starting with \( G_L \). As such, the evaluation only needs to be run until reaching a working point that is acceptable for the intended use-case.
\end{itemize}
Further discussion of the performance-complexity trade-off problem is considered in \cite{prochazka_downstream_2022}.
