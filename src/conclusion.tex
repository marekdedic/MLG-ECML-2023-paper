\section{Conclusion}

In this work, an extension of the HARP algorithm was proposed, which generalizes it from a method for pretraining to a general graph reduction framework. A novel approach to prolonging graphs in the HARP setting was presented that selectively prolongs the graph in a way that maximizes performance of the considered downstream task under limited graph size. All of the proposed methods were experimentally verified, with the headline result being that at about 40\% reduction in node count, the accuracy was still reasonably close to the accuracy on a full graph for most datasets.

In future work, a direct way of tackling the outlined problem may be explored, along the lines of our preliminary exploration in this direction \cite{prochazka_downstream_2022}. The proposed approach constitutes of only a single coarsening pass, in contrast to the double procedure of coarsening followed by prolongation used in this work. As in this work, the coarsening procedure is viewed as a sequence of edge contractions. This sequence is, however, determined by the performance of an auxiliary linear regression model, providing a more realistic heuristic for the optimal way of gradually decreasing graph complexity.
